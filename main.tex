\documentclass[12pt,a4paper]{article}

% Packages
\usepackage[utf8]{inputenc} 
\usepackage{amssymb}
\usepackage[shortlabels]{enumitem}
\usepackage{faktor} 
\usepackage{fancyhdr}
\usepackage{todonotes}
\usepackage{amsmath}
\usepackage{hyperref}
% \usepackage[capitalize,nameinlink]{cleveref}
\usepackage{amsthm}
% \usepackage[backend=bibtex,style=verbose-trad2]{biblatex}


% Theorem Environments
\theoremstyle{nicestyle}
\newtheorem{theorem}{Theorem}[section]
\providecommand*{\theoremautorefname}{Theorem}
\newtheorem{exercise}[theorem]{Exercise}
\providecommand*{\exerciseautorefname}{Exercise}
\newtheorem{definition}[theorem]{Definition}
\providecommand*{\definitionautorefname}{Definition}
\newtheorem{lemma}[theorem]{Lemma}
\providecommand*{\lemmaautorefname}{Lemma}
\newtheorem{proposition}[theorem]{Proposition}
\providecommand*{\propositionautorefname}{Propositon}
\newtheorem{corollary}[theorem]{Corollary}
\providecommand*{\corollaryautorefname}{Corollary}
\newtheorem{claim}[theorem]{Claim}
\providecommand*{\claimautorefname}{Claim}
\newtheorem{subclaim}[theorem]{Subclaim}
\providecommand*{\subclaimautorefname}{Subclaim}
\newtheorem{convention}[theorem]{Convention}
\providecommand*{\conventionautorefname}{Convention}
\newtheorem{remark}[theorem]{Remark}
\providecommand*{\remarkautorefname}{Remark}
\newtheorem{fact}[theorem]{Fact}
\providecommand*{\factautorefname}{Fact}
\newtheorem{example}[theorem]{Example}
\providecommand*{\exampleautorefname}{Example}
\newtheorem{notation}[theorem]{Notation}
\providecommand*{\notationautorefname}{Notation}
\newtheorem{question}[theorem]{Question}
\providecommand*{\questionautorefname}{Question}

\newtheorem*{exercise*}{Exercise}
\newtheorem*{theorem*}{Theorem}
\newtheorem*{lemma*}{Lemma}
\newtheorem*{proposition*}{Proposition}
\newtheorem*{corollary*}{Corollary}
\newtheorem*{claim*}{Claim} 
\newtheorem*{subclaim*}{Subclaim}
\newtheorem*{convention*}{Convention}

% Proofblack
\newenvironment{proofblack}{\begin{proof}}
  {\renewcommand{\qedsymbol}{$\blacksquare$}\end{proof}}

% Math Operators
\DeclareMathOperator{\card}{card}
\DeclareMathOperator{\Col}{Col}
\DeclareMathOperator{\dom}{dom}
\DeclareMathOperator{\HC}{HC}
\DeclareMathOperator{\ran}{ran}
\DeclareMathOperator{\rank}{rank} 
\DeclareMathOperator{\supp}{supp}
\DeclareMathOperator{\ord}{Ord}
\DeclareMathOperator{\limit}{Lim} 
\DeclareMathOperator{\ZFC}{ZFC}
\DeclareMathOperator{\zf}{ZF} 
\DeclareMathOperator{\tc}{tc}
\DeclareMathOperator{\rk}{rk} 
\DeclareMathOperator{\cf}{cf}
\DeclareMathOperator{\id}{id}
\DeclareMathOperator{\fn}{Fn}
\DeclareMathOperator{\ult}{Ult}
\DeclareMathOperator{\rS}{r \Sigma}
\DeclareMathOperator{\crit}{crit}
\DeclareMathOperator{\trcl}{trcl}
\DeclareMathOperator{\lh}{lh}
\DeclareMathOperator{\wfp}{wfp}
\DeclareMathOperator{\hull}{Hull}
\DeclareMathOperator{\otp}{otp}
\DeclareMathOperator{\pr}{pr}
\DeclareMathOperator{\lex}{lex}
\DeclareMathOperator{\length}{lh}
\DeclareMathOperator{\gch}{GCH}
\DeclareMathOperator{\rud}{rud}
\DeclareMathOperator{\Lp}{Lp}

\begin{document}
\author{Stefan Mesken}
\title{Notes on ``$\mathrm{PFA}$ implies $\mathrm{AD}^{L(\mathbb{R})}$''}
\maketitle

\section{Quick Reference \S 1}
\begin{enumerate}
\item $\kappa$ is a singular, strong limit cardinal such that $\square_{\kappa}$ fails. 
\item $A_0 \subseteq \kappa$ codes $V_{\kappa}$.
\item $\lambda = \kappa^{+ \Lp(A_{0})}$.
\item $\cf(\lambda) < \mu < \kappa$ and $\mu^{\omega} = \mu$.
\item $g$ is a $\Col(\omega,\mu)$-generic filter.
\item Given $U \subseteq \mathbb{R}^g$ and $k < \omega$ a coarse
  $(k, U)$-Woodin mouse (witnessed by
  $S, T, \Sigma, \delta_0, \ldots, \delta_k$) is a countable
  transitive model of $\ZFC$ such that
  \begin{enumerate}
  \item
    $N \models \delta_0 < \ldots < \delta_k \text{ are Woodin
      cardinals}$ ,
  \item
    $N \models S,T \text{ are trees which project to complements after
      the collapse } \\ \text{of } \delta_k \text{ to be countable}$ and
  \item there is a $\omega_1+1$-iteration strategy $\Sigma$ for $N$
    such that whenever $i \colon N \to P$ is an iteration map by
    $\Sigma$ and $P$ is countable, then $p[i(S)] \subseteq U$ and
    $p[i(T)] \subseteq \mathbb{R}^g \setminus U$.
  \end{enumerate}
\item For a norm $\phi \colon \dom(\phi) \to \ord$ we let $\le_{\phi}$
  be the associated prewellorder given by $x \le_{\phi} y$ iff
  $\phi(x) \le \phi(y)$ and similar for $<_{\phi}$.
\item For a scale $\vec{\phi}$ we let $\vec{\phi}^{*}$ be the
  associated sequence of prewellorderes.
\item $(W_{\alpha}^{*})$ Let $U \subseteq \mathbb{R}^g$ and suppose
  there are scales $\vec{\phi}, \vec{\psi}$ on $U$ and
  $\mathbb{R}^g \setminus U$ such that
  $\vec{\phi}^{*}, \vec{\psi}^{*} \in
  J_{\alpha}(\mathbb{R}^{g})$. Then for all $k < \omega$ and all
  $x \in \mathbb{R}^g$ there are $N, \Sigma$ such that
  \begin{enumerate}
  \item $x \in N$ and $N$ is a coarse $(k,U)$-Woodin mouse as
    witnessed by $\Sigma$ and
  \item $\Sigma \restriction \HC^{V[g]} \in J_{\alpha}(\mathbb{R}^{g})$.g
  \end{enumerate}
\item If $W_{\alpha}^{*}$ holds, then
  $J_{\alpha}(\mathbb{R}^{g}) \models \mathrm{AD}$.
\item $\mathcal{L}(C)$ is the language of set theory expanded
  by a constant symbol for each $c \in C$.
\item For any $\Sigma_1$-formula $\theta(v)$ we associate
  $(\theta^{k}(v) \mid k < \omega)$ such that $\theta^k$ is $\Sigma_k$
  and for any $\gamma \in \ord$ and any $x \in \mathbb{R}$
  \[
    J_{\gamma+1}(\mathbb{R}) \models \theta[x] \iff \exists k < \omega
    J_{\gamma}(\mathbb{R}) \models \theta^k[x].
  \]

\item Given a $\Sigma_1^{\mathcal{L}(\{\mathbb{R})\}}$ formula $\theta(v)$
  and $z \in \mathbb{R}$, a $\langle \theta, z \rangle$-witness is a
  $\omega$-sound, $(\omega, \omega_{1}, \omega_1+1)$-iterable
  $z$-mouse $\mathcal{N}$ for which there are
  $\delta_0, \ldots, \delta_9, S, T$ such that $\mathcal{N}$ models
  \begin{enumerate}
  \item $\ZFC$,
  \item $\delta_0 < \delta_1 < \ldots, \delta_9$ are Woodin,
  \item $S, T$ are trees on some $\omega \times \theta$,
    $\theta < \delta_9$ which are absolutely complementing in
    $V^{\Col(\omega, \delta_{9})}$ and
  \item For some $k < \omega$, $p[T]$ is the
    $\Sigma_{k+3}^{\mathbb{R}}$-theory of $J_{\gamma}(\mathbb{R})$
    with $\gamma$ minimal such that
    $J_{\gamma}(\mathbb{R}) \models \theta^k[z]$.
  \end{enumerate}
\item If there is a $\langle \theta, z \rangle$-witness, then
  $L(\mathbb{R}) \models \theta[z]$.
\item $(W_{\alpha})$ If $\theta(v)$ is a $\Sigma_{1}$-formula,
  $z \in \mathbb{R}^g$ and
  $J_{\alpha}(\mathbb{R}^{g}) \models \theta[z]$, then there is a
  $\langle \theta, z \rangle$-witness $\mathcal{N}$ whose iteration
  strategy $\Sigma$ satisfies
  $\Sigma \restriction \mathrm{HC} \in J_{\alpha}(\mathbb{R}^g)$.
\item If $\alpha$ is a limit ordinal, then
  $W_{\alpha}^{*} \implies W_{\alpha}$.
\item An ordinal $\beta$ is critical iff there is some
  $U \subseteq \mathbb{R}^g$ such that $U$ and
  $\mathbb{R}^g \setminus U$ admit scales in
  $J_{\beta+1}(\mathbb{R}^{g})$ but $U$ admits no scale in
  $J_{\beta}(\mathbb{R}^{g})$.
\item If $\beta$ is critical, then $\beta+1$ is critical. If $\beta$
  is a limit of critical ordinals, then $\beta$ is critical iff
  $J_{\beta}(\mathbb{R}^g)$ is not admissible.
\item Let $\beta$ be critial. Then one of the following cases holds true:
  \begin{enumerate}[(1)]
  \item $\beta = \eta + 1$ for some critical $\eta$,
  \item $\beta$ is a limit of critical ordinals and either
    \begin{enumerate}[(a)]
    \item $\cf(\beta) = \omega$ or
    \item The ``inadmissible case'': $\cf(\beta) > \omega$ but
      $J_{\beta}(\mathbb{R}^{g})$ is not admissible.
    \end{enumerate}
  \item The ``end-of-gap case'':
    $\alpha = \sup(\{ \eta < \beta \mid \eta \text{ is critical} \}) <
    \beta$ and either
    \begin{enumerate}[(a)]
    \item $[\alpha, \beta]$ is a $\Sigma_1$ gap or
    \item $\beta-1$ exists and $[\alpha, \beta-1]$ is a $\Sigma_1$
      gap.
    \end{enumerate}
  \end{enumerate}
\item A self-justifying system is a countable set
  $\mathcal{A} \subseteq \mathcal{P}(\mathbb{R})$ which is closed
  under complements in $\mathbb{R}$ and such that every
  $A \in \mathcal{A}$ admits a scale
  $\vec{\psi} = (\psi_{i} \mid i < \omega)$ such that
  $\le_{\phi} \in \mathcal{A}$ for all $i < \omega$. Here we identify
  $\le_{\phi} \subseteq \mathbb{R} \times \mathbb{R}$ with its code
  $\{ x \oplus y \mid x \le_{\psi_{i}} y \} \subseteq \mathbb{R}$.
\end{enumerate}

\subsection{The inadmissible case.}

\begin{enumerate}
\item $\kappa$ is a singular strong limit cardinal such that
  $\kappa^{+ \Lp(A)} < \kappa^+$ for all bounded
  $A \subseteq \kappa^+$ and
  $\mu = \cf(\kappa^{+ \Lp(A_{0})})^{\omega}$ where
  $A_0 \subseteq \kappa$ codes $V_{\kappa}$.
\item to be continued...
\end{enumerate}
% \bibliographystyle{alpha}
% \bibliography{mybib}
\end{document}