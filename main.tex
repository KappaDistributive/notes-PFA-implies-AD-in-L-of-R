\documentclass[12pt,a4paper]{article}

% Packages
\usepackage[utf8]{inputenc} 
\usepackage{amssymb}
\usepackage{faktor} 
\usepackage{fancyhdr}
\usepackage{todonotes}
\usepackage{amsmath}
\usepackage{hyperref}
% \usepackage[capitalize,nameinlink]{cleveref}
\usepackage{amsthm}
% \usepackage[backend=bibtex,style=verbose-trad2]{biblatex}


% Theorem Environments
\theoremstyle{nicestyle}
\newtheorem{theorem}{Theorem}[section]
\providecommand*{\theoremautorefname}{Theorem}
\newtheorem{exercise}[theorem]{Exercise}
\providecommand*{\exerciseautorefname}{Exercise}
\newtheorem{definition}[theorem]{Definition}
\providecommand*{\definitionautorefname}{Definition}
\newtheorem{lemma}[theorem]{Lemma}
\providecommand*{\lemmaautorefname}{Lemma}
\newtheorem{proposition}[theorem]{Proposition}
\providecommand*{\propositionautorefname}{Propositon}
\newtheorem{corollary}[theorem]{Corollary}
\providecommand*{\corollaryautorefname}{Corollary}
\newtheorem{claim}[theorem]{Claim}
\providecommand*{\claimautorefname}{Claim}
\newtheorem{subclaim}[theorem]{Subclaim}
\providecommand*{\subclaimautorefname}{Subclaim}
\newtheorem{convention}[theorem]{Convention}
\providecommand*{\conventionautorefname}{Convention}
\newtheorem{remark}[theorem]{Remark}
\providecommand*{\remarkautorefname}{Remark}
\newtheorem{fact}[theorem]{Fact}
\providecommand*{\factautorefname}{Fact}
\newtheorem{example}[theorem]{Example}
\providecommand*{\exampleautorefname}{Example}
\newtheorem{notation}[theorem]{Notation}
\providecommand*{\notationautorefname}{Notation}
\newtheorem{question}[theorem]{Question}
\providecommand*{\questionautorefname}{Question}

\newtheorem*{exercise*}{Exercise}
\newtheorem*{theorem*}{Theorem}
\newtheorem*{lemma*}{Lemma}
\newtheorem*{proposition*}{Proposition}
\newtheorem*{corollary*}{Corollary}
\newtheorem*{claim*}{Claim} 
\newtheorem*{subclaim*}{Subclaim}
\newtheorem*{convention*}{Convention}

% Proofblack
\newenvironment{proofblack}{\begin{proof}}
  {\renewcommand{\qedsymbol}{$\blacksquare$}\end{proof}}

% Math Operators
\DeclareMathOperator{\card}{card}
\DeclareMathOperator{\Col}{Col}
\DeclareMathOperator{\dom}{dom}
\DeclareMathOperator{\HC}{HC}
\DeclareMathOperator{\ran}{ran}
\DeclareMathOperator{\rank}{rank} 
\DeclareMathOperator{\supp}{supp}
\DeclareMathOperator{\ord}{Ord}
\DeclareMathOperator{\limit}{Lim} 
\DeclareMathOperator{\zfc}{ZFC}
\DeclareMathOperator{\zf}{ZF} 
\DeclareMathOperator{\tc}{tc}
\DeclareMathOperator{\rk}{rk} 
\DeclareMathOperator{\cf}{cf}
\DeclareMathOperator{\id}{id}
\DeclareMathOperator{\fn}{Fn}
\DeclareMathOperator{\ult}{Ult}
\DeclareMathOperator{\rS}{r \Sigma}
\DeclareMathOperator{\crit}{crit}
\DeclareMathOperator{\trcl}{trcl}
\DeclareMathOperator{\lh}{lh}
\DeclareMathOperator{\wfp}{wfp}
\DeclareMathOperator{\hull}{Hull}
\DeclareMathOperator{\otp}{otp}
\DeclareMathOperator{\pr}{pr}
\DeclareMathOperator{\lex}{lex}
\DeclareMathOperator{\length}{lh}
\DeclareMathOperator{\gch}{GCH}
\DeclareMathOperator{\rud}{rud}
\DeclareMathOperator{\Lp}{Lp}

\begin{document}
\author{Stefan Mesken}
\title{Notes on ``$\mathrm{PFA}$ implies $\mathrm{AD}^{L(\mathbb{R})}$''}
\maketitle

\section{Quick Reference \S 1}
\begin{enumerate}
\item $\kappa$ is a singular, strong limit cardinal such that $\square_{\kappa}$ fails. 
\item $A_0 \subseteq \kappa$ codes $V_{\kappa}$.
\item $\lambda = \kappa^{+ \Lp(A_{0})}$.
\item $\cf(\lambda) < \mu < \kappa$ and $\mu^{\omega} = \mu$.
\item $g$ is a $\Col(\omega,\mu)$-generic filter.
\item Given $U \subseteq \mathbb{R}^g$ and $k < \omega$ a coarse
  $(k, U)$-Woodin mouse (witnessed by
  $S, T, \Sigma, \delta_0, \ldots, \delta_k$) is a countable
  transitive model of $\zfc$ such that
  \begin{enumerate}
  \item
    $N \models \delta_0 < \ldots < \delta_k \text{ are Woodin
      cardinals}$ ,
  \item
    $N \models S,T \text{ are trees which project to complements after
      the collapse } \\ \text{of } \delta_k \text{ to be countable}$ and
  \item there is a $\omega_1+1$-iteration strategy $\Sigma$ for $N$
    such that whenever $i \colon N \to P$ is an iteration map by
    $\Sigma$ and $P$ is countable, then $p[i(S)] \subseteq U$ and
    $p[i(T)] \subseteq \mathbb{R}^g \setminus U$.
  \end{enumerate}
\item For a scale $\vec{\phi}$ we let $\vec{\phi}^{*}$ be the
  associated sequence of prewellorderes.
\item $(W_{\alpha}^{*})$ Let $U \subseteq \mathbb{R}^g$ and suppose
  there are scales $\vec{\phi}, \vec{\psi}$ on $U$ and
  $\mathbb{R}^g \setminus U$ such that
  $\vec{\phi}^{*}, \vec{\psi}^{*} \in
  J_{\alpha}(\mathbb{R}^{g})$. Then for all $k < \omega$ and all
  $x \in \mathbb{R}^g$ there are $N, \Sigma$ such that
  \begin{enumerate}
  \item $x \in N$ and $N$ is a coarse $(k,U)$-Woodin mouse as
    witnessed by $\Sigma$ and
  \item $\Sigma \restriction \HC^{V[g]} \in J_{\alpha}(\mathbb{R}^{g})$.g
  \end{enumerate}

\end{enumerate}

% \bibliographystyle{alpha}
% \bibliography{mybib}
\end{document}